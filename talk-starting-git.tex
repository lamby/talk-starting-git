%
% "Git Tutorial"
% Copyright (C) 2007  Chris Lamb <chris@chris-lamb.co.uk
% Based on a template (C) Daniel Watkins <D.M.Watkins@warwick.ac.uk>
%                         Chris Lamb <chris@chris-lamb.co.uk>
%
%  This program is free software; you can redistribute it and/or modify
%  it under the terms of the GNU General Public License as published by
%  the Free Software Foundation; either version 2 of the License, or
%  (at your option) any later version.
%
%  This program is distributed in the hope that it will be useful,
%  but WITHOUT ANY WARRANTY; without even the implied warranty of
%  MERCHANTABILITY or FITNESS FOR A PARTICULAR PURPOSE.  See the
%  GNU General Public License for more details.
%
%  You should have received a copy of the GNU General Public License
%  along with this program; if not, write to the Free Software
%  Foundation, Inc., 51 Franklin St, Fifth Floor, Boston, MA  02110-1301  USA

\documentclass{beamer}

\usepackage{beamerthemesplit}
\usepackage{url} 

\title{First steps with Git}
\author[Chris Lamb, WUGLUG]{Chris Lamb\\Warwick University GNU/Linux User Group\\\small{\url{http://www.wuglug.org.uk/}}} % Replace <AUTHOR> with your name
\date{31st October 2007
\newline
\newline
\tiny{The \LaTeX{} source code for this presentation is licensed under the GNU General Public License.}}

\begin{document}

\frame{\titlepage}

%%%%%%%%%%%%%%%%%%%%%%%%%%%%%%%%%%%%%%%%%%%%%%%%%%%%%%%%%%%%%%%%%%%%%%%%%%%%%%%%%%%%%%%%%%
\section{Preliminaries}

\subsection{What is Git?}
\frame
{
    \frametitle{What is Git?}

    \begin{quote}
        Git is a fast, scalable, distributed revision control system with an
        unusually rich command set that provides both high-level operations and
        full access to internals.
    \end{quote}
}

\subsection{Installing Git}
\begin{frame}[fragile]
    \frametitle{Installing Git}

    \begin{description}
        \item[Debian] \verb/aptitude install git-core/
        \item[OpenSUSE] \verb/zypper install git-core/
        \item[Fedora] \verb/yum install git-core/
        \item[Gentoo] \verb/emerge -va git-core/
        \item[Codd] Already installed.
        \item[DCS] Get source from \url{http://git.or.cz/}, then: \\ 
            \verb# $ ./configure --prefix=$HOME# \\
            \verb# $ make -j 30# \\
            \verb# $ make install#
        \item[MS-Windows] Don't care, YDIW.
    \end{description}
\end{frame}

%%%%%%%%%%%%%%%%%%%%%%%%%%%%%%%%%%%%%%%%%%%%%%%%%%%%%%%%%%%%%%%%%%%%%%%%%%%%%%%%%%%%%%%%%%

\section{Doing stuff with Git}

\begin{frame}[fragile]
    What not to do at this point:

    \begin{verbatim}
        $ git-<TAB>
        Display all 137 possibilities? (y or n)
    \end{verbatim}

    DON'T PANIC.

    \begin{quote}
    Git is a fast, scalable, distributed revision control system with an
    unusually \alert{rich command set that provides both high-level operations and
    full access to internals}.
    \end{quote}

\end{frame}

\begin{frame}[fragile]
    \frametitle{Creating a repository}

    \begin{verbatim}
$ git init
Initialized empty Git repository in .git/
    \end{verbatim}

    \begin{verbatim}
$ ls -la
drwx------  3 lamby lamby 4096 2007-10-28 00:49 .
drwxr-xr-x 74 lamby lamby 4096 2007-10-28 00:49 ..
drwxr-xr-x  7 lamby lamby 4096 2007-10-28 00:49 .git
    \end{verbatim}
\end{frame}

\begin{frame}[fragile]
    \frametitle{Your first commit}

    Committing is simple:

    \begin{verbatim}
$ vim hello.py
$ git add hello.py
$ git commit
Created initial commit 0b7acf6: Intiial commit
 1 files changed, 3 insertions(+), 0 deletions(-)
 create mode 100644 hello.py
$
    \end{verbatim}

\end{frame}

\begin{frame}[fragile]
    \frametitle{What just happened?}

    \begin{itemize}
        \item Git has a \emph{staging area} for commits
        \item \alert{git add} adds files to the staging area
        \item \alert{git commit} commits the staging area
        \item So faaast
    \end{itemize}

    To examine the difference between your , use \alert{git diff}.

    \verb#git commit -a#

\end{frame}


\begin{frame}[fragile]
    \frametitle{Viewing history}

    \begin{itemize}
        \item \alert{git log}
    \end{itemize}
\end{frame}



\begin{frame}[fragile]
    \frametitle{Branches}

    \begin{itemize}
        \item A single Git repository can maintain multiple branches
        \item To create a new branch, run: \verb#git branch new-branch-name#
        \item \verb#git branch# lists current branches:

    \begin{verbatim}
  ~/local/git/linux-linus2.6$ git branch
    lamby-nek4k
    liyu-nek4k
  * master
    nek4k\end{verbatim}
        \item Switch branches with \verb#git checkout branch-name#
    \end{itemize}

\end{frame}



%%%%%%%%%%%%%%%%%%%%%%%%%%%%%%%%%%%%%%%%%%%%%%%%%%%%%%%%%%%%%%%%%%%%%%%%%%%%%%%%%%%%%%%%%%

\section{Sharing your work}

\begin{frame}[fragile]
    First, set your username.

    \begin{verbatim}
$ git config --global user.name "Chris Lamb"
$ git config --global user.email chris@chris-lamb.co.uk
    \end{verbatim}
\end{frame}

\begin{frame}[fragile]
    No master servers - we another repository to 'push' to.

    \begin{description}
        \item[ssh] - Default
        \item[git] - Custom protocol on \verb#9418/tcp#. Efficient. Firewall issues.
        \item[http] - Slow, requires some extra configuration. Firewall friendly.
        \item[rsync] - Fairly efficient.
    \end{description}

\end{frame}

\begin{frame}[fragile]
    Sharing over HTTP.
    
    Create your bare Git repository in \verb#public_html#, or similar.
    Or symlink your repositories \verb#.git# somewhere accessible

    You will need to run \verb#git-update-server-info# if you want 

    Or make this automatic with:

    \begin{verbatim}
  $ chmod +x hooks/post-update
    \end{verbatim}
\end{frame}

\begin{frame}[fragile]
    \begin{verbatim}
    \end{verbatim}
\end{frame}

%%%%%%%%%%%%%%%%%%%%%%%%%%%%%%%%%%%%%%%%%%%%%%%%%%%%%%%%%%%%%%%%%%%%%%%%%%%%%%%%%%%%%%%%%%

\section{Cool stuff}

% Speed (wrt. workflow, garbage collecting)
% CVS server emulation,
% Cheap branching, cheap merging (octopus merge)
% gitk
% Importing SVN repositories
% Interacting with SVN repositories
% Importing from \$TOOL
% Putting the current branch in your PROMPT!

%%%%%%%%%%%%%%%%%%%%%%%%%%%%%%%%%%%%%%%%%%%%%%%%%%%%%%%%%%%%%%%%%%%%%%%%%%%%%%%%%%%%%%%%%%

\frame
{
    \frametitle{Thanks!}
    WUGLUG contact information:
    \begin{itemize}
        \item Website: \url{http://www.wuglug.org.uk}
        \item IRC: {\tt \#wuglug} on {\tt irc.uwcs.co.uk:6667}
        \item Mailing list: \url{https://www.warwickcompsoc.co.uk/mailman/listinfo/wuglug}
    \end{itemize}
}

\end{document}
